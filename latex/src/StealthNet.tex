\documentclass[a4paper,11pt]{article}

\usepackage{StealthNet}

% Graphics directory
\ifpdf
    % pdflatex requires bitmap images.
    \graphicspath{{./img/png/}}
\else
    % latex requires vector images.
    \graphicspath{{./img/eps/}}
\fi

% Title page details
\title{StealthNet Security Implementation}
\author{Joshua Spence \and Ahmad Al Mutawa}
\date{April 2012}

%%%%%%%%%%%%%%%%%%%%%%%%%%%%%%%%%%%%%%%%%%%%%%%%%%%%%%%%%%%%%%%%%%%%%%%%%%%%%%%%
\begin{document}

\maketitle

% INTRODUCTION
\section{Introduction}
This document outlines the modifications that were made to the \packageName{} in
order to secure the communications provided by the application. The provided
source code for the \packageName{} package is vulnerable to a wide range of 
\singleQuote{attacks}, including but not limited to the following.

\begin{description}

\item[Eavesdropping] The communications implemented in \packageName{} transmit
messages in cleartext over sockets. This allows an attacker who has gained 
access to data paths in the network to eavesdrop on \packageName{} 
communications.

\item[Data Modification] As a follow-on from \attackName{eavesdropping}, an 
attacker would additionally be able to alter the transmitted data. An attacker
can modify the data in the packet without the knowledge of the sender or 
receiver.

\item[Denial-of-Service] A \attackName{denial-of-service} attack prevents normal
of the \serviceName{} service. An attacker could overwhelm the \serviceName{} 
server such that it is unable to process legitimate connections from legitimate 
users. Alternatively, an attacker could simply block network traffic completely,
resulting in a complete lack of communication between \serviceName{} peers.

\item[Man-in-the-Middle] A \attackName{man-in-the-middle} attack occurs when 
someone between two \serviceName{} peers is actively monitoring, capturing, and 
controlling your communication transparently. An attacker makes independent 
connections with the victims and relays messages between them, making them 
believe that they are talking directly to each other over a private connection.

\end{description}

% AUTHENTICATION
\section{Authentication}

\subsection{Introduction}

\subsection{Details}
Diffie-Hellman key exchanges involves combining your private key with your 
partner's public key to generate a number. The peer does the same with its 
private key and our public key. Through the magic of Diffie-Hellman we both come
up with the same number. This number is secret (discounting MITM attacks) and 
hence called the shared secret. It has the same length as the modulus, e.g. 512 
or 1024 bit. Man-in-the-middle attacks are typically countered by an independent
authentication step using certificates (RSA, DSA, etc.).

The thing to note is that the shared secret is constant for two partners with 
constant private keys. This is often not what we want, which is why it is 
generally a good idea to create a new private key for each session. Generating a
private key involves one modular exponentiation assuming suitable Diffie-Hellman
parameters are available.

In TLS the server chooses the parameter values itself, the client must use those
sent to it by the server.

The use of ephemeral keys as described above also achieves what is called 
``forward secrecy''. This means that even if the authentication keys are broken 
at a later date, the shared secret remains secure. The session is compromised 
only if the authentication keys are already broken at the time the key exchange 
takes place and an active MITM attack is used. This is in contrast to 
straightforward encrypting RSA key exchanges.

The protocol depends on the discrete logarithm problem for its security. It 
assumes that it is computationally infeasible to calculate the shared secret key
$k = generator^{ab} \bmod prime$ given the two public values 
$generator^{a} \bmod prime$ and $generator^{b} \bmod prime$ when the prime is 
sufficiently large. Maurer has shown that breaking the Diffie-Hellman protocol 
is equivalent to computing discrete logarithms under certain assumptions.

The Diffie-Hellman key exchange is vulnerable to a man-in-the-middle attack. In 
this attack, an opponent Carol intercepts Alice's public value and sends her own
public value to Bob. When Bob transmits his public value, Carol substitutes it 
with her own and sends it to Alice. Carol and Alice thus agree on one shared key
and Carol and Bob agree on another shared key. After this exchange, Carol simply
decrypts any messages sent out by Alice or Bob, and then reads and possibly 
modifies them before re-encrypting with the appropriate key and transmitting 
them to the other party. This vulnerability is present because Diffie-Hellman 
key exchange does not authenticate the participants. Possible solutions include 
the use of digital signatures and other protocol variants.

\subsection{Implementation}

\subsection{Outstanding vulnerabilities}
 
% ENCRYPTION
\section{Encryption}

\subsection{Introduction}

\subsection{Details}

\subsection{Implementation}

\subsection{Outstanding vulnerabilities}

% INTEGRITY
\section{Integrity}

\subsection{Introduction}

\subsection{Details}

\subsection{Implementation}

\subsection{Outstanding vulnerabilities}

% REPLAY PREVENTION
\section{Replay prevention}

\subsection{Introduction}

\subsection{Details}

\subsection{Implementation}

\subsection{Outstanding vulnerabilities}

% CONCLUSION
\section{Conclusion}

\end{document}
